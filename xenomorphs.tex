\documentclass{article}
\usepackage{csquotes}

\usepackage[natbib,backend=biber,style=apa]{biblatex}
\addbibresource{xenomorphs.bib}
\title{Xenomorphs as Nominalists}
\author{Malin Freeborn}
\date{2019-09-02}

\DeclareLabelname[movie]{
      \field{writer}
      \field{director}
    }

\DeclareLabelname[video]{
      \field{writer}
    }

\begin{document}

\maketitle

\tableofcontents

\begin{abstract}

Two strong tendencies appear throughout science fiction films. Less serious, family-friendly, often-comedic (`fluffly'), sci-fi necessitates at all times a metaphysical realist outlook of some form, whether in the form of Ethical Realism, Mental Realism, or some other type. On the far side of the fence, horror sci-fi rarely or never presents any such necessity for metaphysical realism, and more often than not poses direct challenges to it, often in accordance with normal nominalist arguments. This essay provides examples -- some of which come directly from speeches within the films, but more often are displayed through an analysis of metaphysical necessity.  

	\textit{Keywords:} metaphysical realism, horror, science fiction, sci-fi, nominalism, film analysis.

\end{abstract}

\pagebreak

Two strong tendencies run through sci-fi films, like good and evil twins, admitting of few or no exceptions. On the one hand, the fluffier, more family-friendly sci-fi necessitates some part of a metaphysically realist conception of the universe, while horror sci-fi such as Alien and Predator contain no such constraint, and further, the better quality horror sci-fi has a strong tendency to pose direct challenges to Realist notions of one sort or another.

While `horror sci-fi' is a fairly self-explanatory category, I should spend a moment cutting out exactly what I intend by ``fluffy sci-fi''. If criticising the realism of a film feels ridiculous due to giving too much gravity to a light-hearted work, then this is fluffy sci-fi. Many feel comfortable saying that \textit{The Matrix} fails its audience when it claims that human bodyheat could be used as a powersource because it cuts the immersion of the film, but if we attempt to criticise \textit{Thor} for its unrealistic physics when the titular character throws his hammer Mjolnir, then we can risk looking ridiculous by taking it `too seriously'. \textit{Thor} is not meant to withstand probing analysis, so we must count it as fluffy sci-fi.  

Any sci-fi neither ``fluffy'' nor horror falls outside the scope of this hypothesis.

\section{The Realist Outlook}
\subsection{Price \& Paul}
Metaphysical Realists such as Price work to establish a few points. Chief among them in Price's work are:\footcite[p 7-9]{thinkingexperience}

\label{realistgoals}
\begin{enumerate}
\item how properties can seem to be in different things at once,
\item how things can be partly similar,
\item how our language can accurately reference the world.
\end{enumerate}
Price seems to be reaffirming our belief in our worldview. He does not challenge the similarities we see, or how we categorise the world, but demands a world-view which pays homage to them and seeks to make this worldview sharper.

These points will be returned to later, as films such as Predator pose challenges to them. But for now, I want to talk through how fluffy sci-fi such as \textit{Paul}\footcite{paul} necessitates a realist outlook. Paul the alien reaffirms common, European ways of seeing the world by talking as we do and referencing the same objects. His alienness is fairly mild, consisting in an odd appearance and a propensity to eat live birds. Outside of this, he has similar ideas of good, evil, etiquette, and even has a standard middle-class reactions to theological debates.

However, \textit{Paul}, goes farther than this in reaffirming our worldview by necessitating some kind of metaphysical realist outlook upon the world. During one scene, Paul the alien heals someone's diseased eye by transferring the disease into his own eye and then healing himself. This means that disease can shift from one object to another -- not in the manner of a contagion but as a property transference, meaning the property must have some independent existence outside of the eyeball. The film clearly indicates that the same disease which was once in a woman's eyeball
subsequently resides in Paul's.

Compare this to what little is necessitated by modern science. Modern scientists could currently deny the existence of disease as a proper category without any violations of a single scientific fact. In short, modern scientists may consistently call themselves nominalists. We classify cells in different ways, and it suits our purposes to classify them as `healthy' and `diseased', because these descriptions allow us to work with eyeballs to achieve what we want, but in reality a scientifically-minded person could, without contradiction, state that no disease exists, rather there are only cells in different states and some operate as we wish them to while others do not. In reality (or so a nominalist scientist might claim) no independent property which must be called `disease' exists -- we simply have a convenient label.

To flesh this point out, consider what we have so-far found to be recognised by the universe. One can easily make a device which will recognise gravity -- something which will point to nearby gravitational sources and state how much gravitational force it exerts. This seems to force everyone to say that gravity exists in more than name -- one can hardly operate a sane worldview and deny this. Comparatively, there are no devices which measure morality; nothing to measure justice or possibility, such as when we say something was a missed opportunity; and nothing to detect the existence of a mind or even of memory. Scientists are free to study memory in a nominalist fashion, claiming that this classification works for their purposes, without imagining that the universe will recognise or respect such classifications. In short, salt, gravity, length, and heat are all detectable, so they must exist due to the official stamp the universe has given them. Meanwhile concepts such as `justice' may serve as nothing but a name for an arbitrary collection of patterns.

It is this stamp of approval which \textit{Paul} hands to those who live in his world -- the ability to demonstrate to any nominalists that disease must act upon the body as a property which is detectable and transferable. And evidence that nominalism must be false equates to showing that realism is true.

\subsection{Rick \& Lewis}
Realism takes many different forms, and in \textit{Rick and Morty} we find a sm\"org{\aa}sbord. The show calls Rick a scientist, though really the term `engineer' fits better. He invents a portal gun to travel to alternate realities, marking this world as one in which modal realism holds and is recognised by the universe.\footcite[\textit{Close Rick-counters of the Rick Kind}, s. 1, e. 10]{ricknmorty} This does not precisely follow the works of modal realists such as Divers\footcite[p 217-18]{genuinerealisttheory} and Lewis\footcite[p 70]{pluralityworlds}, as Rick and his assistant Morty travel to these other universes -- something Lewis prohibits in his theory. Nevertheless, the show must exist in some kind of modal realist multiverse.

Once again, the metaphysical realism portrayed distinctly validates the standard European worldview. In one scene we
take a wild ride through the multiverse, looking at all the different ways the world might have formed. We find a world where pizzas order people on their phones, then a world where phones sit on people and order chairs, then a world where chairs sit on people and order phones using pizzas instead of phones. These worlds may seem far-fetched and weird, but all of them are composed strictly of elements of this standard European worldview -- some set number of universals are plucked from the authors' minds and used as the basic ingredients for a variety of soups. The universe (or multiverse) clearly recognises that some things partake in the universal `chair', and then instantiates chairs in a variety of ways in a variety of universes.

\section{The Nominalist Challenge}
\subsection{Mackie \& Giger}
On the other side of the fence, we find sci-fi horror, where no Metaphysical Realism is necessitated, and where -- more often than not -- direct challenges are posed to Realism. Both \textit{Alien} and \textit{Predator} push such challenges and will serve as my poster children, though they also challenge metaphysical realism in subtler ways.

Most commonly, such films challenge ethical realism, just as Mackie does.\footcite[p 38]{inventingrightwrong} One of Mackie's two chief challenges to ethical realism is its Epistemological ``queerness''. This can be thought of easily in terms of impossible bets. I can make bets with people about whether some building is as tall as the Palace of Culture in Warsaw, or whether or not my coffee is over 60 degrees, and this bet will have a definitive winner. These answers are recognised by the universe. But ethical statements are recognised only by people, and as a result no bets will grant payments. I might say that we should ban eating animals, or that we should speak well of our parents, but I cannot make a bet about the truth of such statements.

Such thoughts are echoed in the words of Ash, the ship's robot:

\begin{quotation}
Lambert: You admire it [the alien].

	Ash: I admire its purity. A survivor {\dots} unclouded by conscience, remorse, or delusions of morality.\footcite{alien}

\end{quotation}

Ash tells us that morality is a delusion as a sharp reminder that the xenomorph (the alien) has failed to understand or perceive precisely nothing -- a denial of the realist's third goal,\footnote{See \ref{realistgoals}} that our language accurately reflects the world.  Commonly, people will say of animals which kill that they `don't know any better', but no such excuse can be given for the xenomorph. The creatures seem to have built a spaceship in \textit{Alien}, and in \textit{Aliens} we can see the `queen xenomorph' operates machinery\footcite{aliens}. The aliens do not fail to see ethical problems with murdering people and then using them as hosts because they lack intelligence, nor can we claim that we somehow sense things which they do not. Rather, we see ethical issues where they see none because, as Ash explains, we delude ourselves.

This difference in perception is not limited to the xenomorph's lack of ethics. Its survival dependences seem to require no distinction between humans and dogs. In normal discourse we would create a sharp distinction between ourselves and any dog, and realists must assert that we can justify this distinction by appealing to the number of universals instantiated in the one which is not in the other. However, the xenomorph, with no need of seeing these universals, may simply group us under the heading of `calories'. It is in \textit{$Alien ^{3}$} that we find an alien using a dog as a host, just as it might with any person.\footcite{alien3} The kinds of distinctions we make find themselves undermined by this creature which walks through them without notice, because if the rest of the world does not care about dividing itself into the kinds of universals we perceive, then it suggests they reside only in our minds.

Where \textit{Rick and Morty} understand and validate the divides we see in our world, Alien shows creatures walking blindly through such divisions as they simply do not exist for the xenomorphs.

\subsection{Predator \& Colours}
Realists such as Price often use colours as their go-to example of real properties.\footnote{\cite[p 27]{metaphysicscontemporaryintro} and \cite[p 8]{thinkingexperience}} Many like to point out that this is `red' and that is `green', and that such colours are objectively real features of the object, that things often have the same colour, and that this suggests that some third thing -- the colour itself -- instantiates itself in both objects. Such Realist Metaphysicians would have a difficult time explaining this to a predator alien.

Predators see a higher wavelength in the electromagnetic spectrum than humans, so objects in their minds are coloured according to the amount of heat they give off.\footcite{predator} This not only means that predators have no need of admitting the colour `red' as a we understand it as a universal, it also means that predators can demand new universals to reference the kinds of categories they see. As with xenomorphs, above, predators may well view dogs and humans under the heading of a single word, which would then demand some new universal to reference under a realist worldview. This word might not simply reference humans but any object at 37-degrees temperature, where most humans sit. It could plausibly reference humans, mice, and some warm baths, while banishing dead humans from the class as being less similar to any of these objects than a living human might be from a warm bath.

The danger here for Realists lies in the notion of real similarity being undone -- our third realist goal.\footnote{See \ref{realistgoals}} Our language groups things which are supposed to be similar, but if a limitless number of worldviews could group the world in various ways, then we could not accurately say that things have real similarities, only that we find them similar to live the kinds of life we lead. Our notions of likeness could not claim truth, only usefulness, posing a direct
challenge to goal number 2 of the realists.

Of course, the issues do not stop with one additional way to see the world. Once we admit that the predators' words and corresponding categories have as much validity as ours, we must also admit of a limitless number of other worldviews, each based around different values. The objective similarities of words will vanish into nothing as everything becomes similar or dissimilar to everything else, to every possible degree, so long as one selects the right language from a race with just the priorities to group those things.

\subsection{Predators' Honour}
Predators may at first glance seem to suggest some realist notion of ethics, but further analysis shows that their ethical codes do not align with any of our notions of ethics. In \textit{Predator 2}, the police officer Lieutenant Harrigan fatally wounds one of the predators\footcite{predator2}. More predators surround him, but instead of killing him, they take away their wounded comrade and give the Lieutenant a war-prize as a congratulations. The predators exhibit some kind of honour code, though while at first glance this might seem to reinforce ethical realism, because alien creatures still show some kind of decency, the details show that their ethical code bears no resemblance to our own.

Humans do not kill alone during warfare -- only when hunting. The predators might want to hunt humans, but they refuse to kill unarmed prey -- something no human hunters do. So the predators do not have our ethical attitudes, either from hunting or warfare. Instead, their bizarre ethical system gives us a strong suggestion that while alien races may have ethical codes, the differences in such codes show that there is no one true ethical code. This essentially amounts to the anthropological moral skepticism argument writ large; a reminder that we can extend it beyond current boundaries, and leaving us in a world devoid of any solid ethical foundation but rather a galaxy full of differing ideas, based on nothing but arbitrary cultural needs.

\section{Categories of Realism and a Summary}
Just as a recap, I should mention that I do not propose that metaphysical realists must accept unscientific sci-fi ideas. Rather, any fluffy sci-fi must necessitate a metaphysically realist worldview by showing the universe recognising the kinds of universals which metaphysical realists think we ought to believe in.

The preceding parts have focussed in detail on what I intend by the necessity (or lack of) in sci-fi films. In the next few sections, I want to perform a run-through of more general tendencies with more examples, to show that this tendency extends far beyond the few films so-far analysed.

\subsection{Property Realism}
Realism concerning properties as instantiations of universals shows itself in many ways, but primarily when a universal is transferred or altered, where no actual thing-in-itself could possibly register to a machine according to modern scientific understanding. A recurring example is changes in size, which naturally implies that there is such a thing as size. Of course, some things have more matter than others, but when \textit{Ant-Man}\footcite{ant-man}, or the children in \textit{Honey, I Shrunk the Kids}\footcite{honeyishrunkthekids} change their size, they do not simply lose mass by having most of their bodies removed. Instead, some part of the machine recognised the boundaries of the children in order to shrink them. Were someone in fact to shrink, then they could not breath, because the oxygen atoms could not pass through the small lung membranes due to structural changes or a difference in relative size of the atoms of air and in the child. Of course, none of this can come into effect, because the film simply operates under the assumption that the child gains the universal `small' (or somehow adjusts its `size universal') and that no other effects take place. The notion of a simple collection of atoms having some connection to some very real `size' property, which can be modified, necessitates that at least some properties do exist. Once again, if a physicist were to reject the outdated Aristotelian notion that objects instantiate a particular size, we could -- in such film-universes -- point to Ant-Man or some shrink-ray as positive proof that objects have a particular property called `size', just as surely as they have a particular radiation level.

Transference also clearly displays the existence of something, because there must be something to transfer. Above we covered disease transference with \textit{Paul}, and \textit{Power Rangers}\footcite{powerrangers} seems to operate in a similar way. The Power Rangers obtained their name by transferring `power' from a `power crystal'. Normally we would talk of power as a single concept, but when pressed, most people would admit that this category is composed of our own uses for groups of things, rather than being an objective property. `Power' may refer to electrical joules, or muscles, or political sway when we use this word in a colloquial setting -- an eclectic mixture of concepts. However, within \textit{Power Rangers} the universe reassures us that this concept refers to a real and more unified property, and because `power' really exists, not just as a useful idea but as something one can store in a crystal and leach out to gain `power' in the sense of speed, agility, and strength.

In the \textit{X-Men}\footcite{xmen}, a similar feature can be seen in the character Storm's powers. She can control the electricity of lightning, but not in electrical wires, and control the water in rain or at the sea, but not tap-water, because the universe recognises which things count as `weather' and which do not. Of course, nobody would say that rain does not exist, but outside a world where Storm exists one can doubt that `weather' exists, supposing it only to be a word which bundles together different phenomena which we regard in a similar way. Storm's presence demands that this word is a proper category, referencing some very real property of the things her powers target.

In \textit{Aquaman}\footcite{aquaman}, we find someone who controls marine life, where the property of being `marine' in reality lies in a grey area. Animals tend to inhabit sources closer or farther from the depths of the ocean, but nature has no strict boundaries about whether ducks or platipuses are `marine' or `non-marine'. Yet Aquaman clearly has powers over just this type of creature, showing that there is some definite property here, and one which can be measured. After all, if someone doubted that this was more than a convenient taxonomical word, we could point them to Aquaman's clear ability to dilineate what counts and what does not.

In \textit{Thor}\footcite{thor}, the hammer Mjolnir is granted the ability to tell the worthy from the unworthy. Initially, we might imagine that the hammer uses some Asgardian technology to read minds and from what it reads measure worthiness. But the film clearly shows that in Asgard people can deceive each other easily, meaning they have no access to mind-reading technology. So what might Mjolnir detect?  The film answers only, `worthiness'; and since it can detect worthiness itself, since the universe recognises this property, the property must exist. In such a universe, one is forced to take a metaphysically realist stance about at least one property.

\subsection{Mental Realism}
Fluffy sci-fi, when interacting with anything of the mind, shows a clearly mental realist outlook at every available point. In \textit{Pacific Rim}\footcite{pacificrim} minds transfer from one body to another. In \textit{Flash Gordon}\footcite{flashgordon}, Flash uses a device which detects particular minds and sends intelligible messages to them. This clearly shows as more than simple collections of cells which compute items -- they are a separate category known as `mind' and the universe recognises them as such.

Demeter, in \textit{Two Kinds of Mental Realism}\footcite[p 59-60]{twokindsmentalrealism} makes finer distinctions than most sci-fi, as he notes that most Mental Realists in fact presume the reality of mental states rather than discussing `mind' itself as a real entity. But even such fine distinctions seem apt on occasion for sci-fi. In \textit{Rick and Morty}\footcite[\textit{Rickshank Rickdemption}, s. 3, e. 1]{ricknmorty} we find Rick transfers his mind to an insect-humanoid. He states that the bug does not have ``enough room'' to fit his own mind, so he leaves behind his improv classes and one phobia. This collection of mental states must merit the status of being real entities if Rick can transfer them from one brain to another -- entities distinct from the other thoughts in his head, and not mere names for more complicated mechanisms in the brain as a nominalist might be tempted to say, without any independent reality beyond their utility as a category. Rick's memories of his improv classes exist with real locations which we might measure and change.

\subsection{Ethical Realism}
The existence of real good and evil -- of real properties which we might detect and verify rather than simply talk about -- seems the purview of fantasy novels rather than sci-fi, but fluffy sci-fi as usual breaks down the boundaries suggested by actual science and feels free to tell a tale of a universe where such categories measureably pertain. In \textit{Rick and Morty}, the scientist Rick meets the devil and soon after creates a device which can measure good and evil\footcite[\textit{Something Ricked This Way Comes}, s. 1, e. 9]{ricknmorty}

In \textit{The Fifth Element}, Lilu must destroy Evil itself -- an entity which seems the very platonic form of evil made manifest in the world\footcite{fifthelement}. When the military fire missiles into this massive black orb, it simply grows larger. Father Cornelius explains, \begin{quote}Evil begets evil, Mister President. Shooting will only make it stronger\end{quote}. This thing not only seems to instantiate evil, it is evil itself, so it reacts as the evilness of a deed, growing stronger through violence.

If we consider the film \textit{Alien} at this point, it should feel clear that no `evil detecting device' could be used to hunt a xenomorph, as it would fly in the face of exactly what this kind of film offers -- a cold and indifferent universe, which operates outside of the conceptual confines of the human mind. Similarly, a mind-reading device, allowing people to chat with xenomorphs, would undermine the horror of dealing with an alien mind, and humanise the creature too much to maintain an atmosphere of dread.

\section{Exceptions and Conclusion}

I want to finish this piece with one small potential exception. \textit{Terminator 2: Judgement Day} is not strictly speaking horror, but action sci-fi.\footcite{terminator2} Still, it aims to scare and grants similar push-backs against metaphysical realism to others. The T-100 robot sees reality differently to humans, just as the predator. It kills without hesitation so long as the deaths align with its aims, just as xenomorphs do. Throughout the film, it struggles to understand sorrow or the immorality of murder. Despite this, at the end, the child John Connor cries when the T-100 must self-terminate. The robot looks at him and says \begin{quote}Now I know why you cry\end{quote}\ldots before allowing Sarah Connor to lower it into a vat of molten metal. The film does not tell us exactly what the T-100 understands, but suggests that it understands sorrow -- that there exists some real underlying sadness to particular events which even a robot might come to understand. I propose that the reason the ending of \textit{Terminator 2} feels so poignant rests on this last-minute hint of a switch to a Realist world-view, because at this final juncture the terminator has reaffirmed our value system by joining us in seeing it. The final scene feels tragic, but some otherwise unfeeling part of the universe has recognised this sadness.

\pagebreak

\printbibliography

\end{document}
